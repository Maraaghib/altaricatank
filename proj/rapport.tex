\documentclass[a4paper]{book}
\usepackage{fullpage}

\usepackage[utf8]{inputenc}
\usepackage[T1]{fontenc}
\usepackage[francais]{babel}

\usepackage{latexsym}
\usepackage{fancyhdr}
\usepackage{makeidx}
\usepackage{graphics}
\usepackage{graphicx}
\usepackage{longtable}
\usepackage{moreverb}
\usepackage{listings}

\newcommand{\altarica}{{\sc AltaRica}}

\begin{document}

\title{Master 1, Conceptions Formelles\\
Projet du module \altarica\\
Synthèse (assistée) d'un contrôleur du niveau d'une cuve}

\date{}

\author{Nom1 \and Nom2 \and Nom3}

\maketitle

\chapter{Le sujet}
\section{Cahier des charges}

Le système que l'on souhaite concevoir est composé~:
\begin{itemize}
\item d'un réservoir contenant {\bf toujours} suffisamment d'eau pour alimenter l'exploitation,
\item d'une cuve,
\item de deux canalisations parfaites amont reliant le réservoir à la cuve, et permettant d'amener l'eau à la cuve,
\item d'une canalisation parfaite aval permettant de vider l'eau de la cuve,
\item chaque canalisation est équipée d'une vanne commandable, afin de réguler l'alimentation et la vidange de la cuve,
\item d'un contrôleur.
\end{itemize}

\subsection{Détails techniques}

\subsubsection{La vanne}
Les vannes sont toutes de même type, elles possèdent trois niveaux de débits correspondant à trois diamètres d'ouverture~: 0 correspond à la vanne fermée, 1 au diamètre intermédiaire et 2 à la vanne complètement ouverte. Les vannes sont commandables par les deux instructions {\tt inc} et {\tt dec} qui respectivement augmente et diminue l'ouverture. Malheureusement, la vanne est sujet à défaillance sur sollicitation, auquel cas le système de commande devient inopérant, la vanne est désormais pour toujours avec la même ouverture.

\subsubsection{La Cuve}
Elle est munie de $nbSensors$ capteurs (au moins quatre) situés à $nbSensors$ hauteurs qui permettent de délimiter $nbSensors+1$ zones. La zone 0 est comprise entre le niveau 0 et le niveau du capteur le plus bas; la zone 1 est comprise entre ce premier capteur et le second, et ainsi de suite.

Elle possède en amont un orifice pour la remplir limité à un débit de 4, et en aval un orifice pour la vider limité à un débit de 2.  

\subsubsection{Le contrôleur}
Il commande les vannes avec les objectifs suivants ordonnés par importance~:
\begin{enumerate}
\item Le système ne doit pas se bloquer, et le niveau de la cuve ne doit jamais atteindre les zones 0 ou $nbSensors$.
\item Le débit de la vanne aval doit être le plus important possible.
\end{enumerate}

On fera également l'hypothèse que les commandes ne prennent pas de temps, et qu'entre deux pannes et/ou cycle {\em temporel}, le contrôleur à toujours le temps de donner au moins un ordre. Réciproquement, on fera l'hypothèse que le système à toujours le temps de réagir entre deux commandes.

\subsubsection{Les débits}
Les règles suivantes résument l'évolution du niveau de l'eau dans la cuve~:
\begin{itemize}
\item Si $(amont > aval)$ alors au temps suivant, le niveau aura augmenté d'une unité.
\item Si $(amont < aval)$ alors au temps suivant, le niveau aura baissé d'une unité.
\item Si $(amont = aval = 0)$ alors au temps suivant, le niveau n'aura pas changé.
\item Si $(amont = aval > 0)$ alors au temps suivant, le niveau pourra~:
  \begin{itemize}
  \item avoir augmenté d'une unité,
  \item avoir baissé d'une unité,
  \item être resté le même.
  \end{itemize}
\end{itemize}

\section{L'étude}

\subsection{Rappel méthodologique}
Comme indiqué en cours, le calcul par point fixe du contrôleur est exact, mais l'opération de projection effectuée ensuite peut perdre de l'information et générer un contrôleur qui n'est pas satisfaisant. Plus précisemment, le contrôleur \altarica\ généré~:
\begin{itemize}
\item ne garanti pas la non accessibilité des \emph{Situations Redoutées}.
\item ne garanti pas l'absence de \emph{nouvelles situations de blocages}.
\end{itemize}

Dans le cas ou il existe toujours \emph{des situations de blocages ou redoutées}, vous pouvez au choix~:
\begin{enumerate}
\item Corriger manuellement le contrôleur calculé (sans doute très difficile).
\item Itérer le processus du calcul du contrôleur jusqu'à stabilisation du résultat obtenu. 
  \begin{itemize}
  \item Si le contrôleur obtenu est sans blocage et sans situation redoutée, il est alors correct.
  \item Si le contrôleur obtenu contient toujours des blocages ou des situations redoutées, c'est que le contrôleur initial n'est pas assez performant, mais rien de garanti que l'on soit capable de fournir ce premier contrôleur suffisemment performant.
  \end{itemize}
\end{enumerate}

{\bf Remarque} : Pour vos calculs, vous pouvez utiliser au choix les commandes~:
\begin{itemize}
\item {\tt altarica-studio xxx.alt xxx.spe}
\item {\tt arc -b xxx.alt xxx.spe}
\item {\tt make} pour utiliser le fichier GNUmakefile fourni.
\end{itemize}

\subsection{Le travail a réaliser}

Avant de calculer les contrôleurs, vous devez répondre aux questions suivantes.
\begin{enumerate}
\item Expliquez le rôle de la constante $nbFailures$ et de la contrainte, présente dans le composant {\tt System}, $nbFailures >= (V[0].fail + V[1].fail + V[2].fail)$.
\item Expliquez le rôle du composant {\tt ValveVirtual} et de son utilisation dans le composant {\tt CtrlVV}, afin de remplacer le composant {\tt Ctrl} utilisé en travaux dirigés.
\end{enumerate}

L'étude consiste à étudier le système suivant deux paramètres~:
\begin{enumerate}
\item $nbFailures$~: une constante qui est une borne pour le nombre de vannes pouvant tomber en panne.
\item Le contrôleur initial qui peut être soit {\tt Ctrl}, soit {\tt CtrlVV}.
\end{enumerate}

Pour chacun des huit systèmes étudiés, vous devez décrire votre méthodologie pour calculer les différents contrôleurs et répondre aux questions suivantes~:

\begin{enumerate}
\item Est-il possible de contrôler en évitant les blocages et les situations critiques ?
\item Si oui, donnez quelques caractéristiques de ce contrôleur, si non, expliquez pourquoi.
\item Est-il possible de contrôler en optimisant le débit aval et en évitant les blocages et les situations critiques ?
\item Si oui, donnez quelques caractéristiques de ce contrôleur, si non, expliquez pourquoi.
\end{enumerate}


\chapter{Le rapport}
\section{Rôle de la constante {\tt nbFailures} (2 points)}
La constante nbFailures est utilisée pour limiter le nombre de configurations atteignables. Elle correspond au nombre de valves qui peuvent être en panne en même temps.

$nbFailures >= (V[0].fail + V[1].fail + V[2].fail)$ correspond au fait que le nombre de valve qui tombent en panne doit toujours être inférieur à la valeur de la constante nbFailure, car cette ligne se trouve dans la section $assert$ du code.

\section{Résultats avec le contrôleur initial {\tt Ctrl}}
\subsection{Calcul d'un contrôleur}
\subsubsection{Avec 0 défaillance (1 point)}
\lstinputlisting{Res/System0FCtrl.res}
\lstinputlisting{Res/System0FCtrl0F1I.res}
%\lstinputlisting{Res/System0FCtrl0F2I.res}
%\lstinputlisting{Res/System0FCtrl0F3I.res}
%\lstinputlisting{Res/System0FCtrl0F4I.res}
\paragraph{Interprétation des résultats : Contrôleur sans débit optimisé}
Le controlleur peut controller en évitant les blocages car $deadlock = 0$. Cependant, il ne peut pas assurer qu'on évitera les situations redoutés,
car on peut voir que $SR = 86$, ce qui indique qu'il y a 86 états dans lesquels le controlleurs se trouvera dans une situation redoutée.

\paragraph{Interprétation des résultats : Contrôleur avec débit optimisé}

Lorsqu'on optimise le débit aval, il est possible de contrôller en évitant tout blocages ou toute situation critique. En effet, la variable $SR = 0$.
On peut voir que dans ce contrôleur on a un total de 96 états ($any\_s$) et 858 transitions ($any\_t$). On peut aussi voir, via $CtrlCanControl$, qu'il y a 27 transisitions possibles différentes à partir de l'état initial du contrôleur.

\subsubsection{Avec 1 défaillance (1 point)}
\lstinputlisting{Res/System1FCtrl.res}
\lstinputlisting{Res/System1FCtrl1F1I.res}
%\lstinputlisting{Res/System1FCtrl1F2I.res}
%\lstinputlisting{Res/System1FCtrl1F3I.res}
%\lstinputlisting{Res/System1FCtrl1F4I.res}
\paragraph{Interprétation des résultats : contrôleur sans débit optimisé}
Avec 1 défaillance, on remarque qu'on peut pas contrôler en évitant les situations rédoutées car $SR$ est égale à l'union de $deadlock$ et de $NC$.
Donc dans notre cas, on a $SR$ qui partage les mêmes états avec $NC$.

Conclusion : le controleur peut contrôler car la variable CtrlCanControl est positive, mais on ne peut pas éviter de tomber dans des situations redoutées.

\paragraph{Interprétation des résultats : contrôleur avec débit optimisé}
Avec l'optimisation du débit 'aval', on remarque qu'on a moins d'états à gérer donc c'est plus facile pour le contrôleur de controler.

Cependant, on s'apperçoit dorénavant qu'on peut avoir des blocages à cause de la variable $deadlock = 96$, ce qui revient à dire qu'on ne peut pas contrôler sans éviter de tomber dans des situations redoutées car :

$NC$ et $deadlock$ partagent $69$ états et donc on risque d'augmenter les chances de tomber dans des états bloquants.

Conclusion : Malgré le fait qu'on ait optimisé le débit d'aval, on a constaté qu'on ne pourrait pas contrôler sans éviter de tomber dans $SR$ ou dans $deadlock$.
Aussi, on peut bien voir que $CCoupGagnant$ a beaucoup moins de transitions($2909$).

\subsubsection{Avec 2 défaillances (1 point)}
\lstinputlisting{Res/System2FCtrl.res}
\lstinputlisting{Res/System2FCtrl2F1I.res}
%\lstinputlisting{Res/System2FCtrl2F2I.res}
%\lstinputlisting{Res/System2FCtrl2F3I.res}
%\lstinputlisting{Res/System2FCtrl2F4I.res}
\paragraph{Interprétation des résultats : Contrôleur sans débit optimisé}
Ce controlleur peut aussi s'éxécuter en évitant les blocages car $deadlock = 0$. Cependant, il ne peut pas assurer qu'on évitera les situations redoutés.
Ici, $SR = 551$. On a donc beaucoup d'états dans lesquels on ne souhaite pas aller. Ceci est dû au fait qu'on peut avoir deux valves défaillantes.
Si une d'entre elle est celle de la sortie, on ne peut alors plus vider l'eau de la cuve et on atteint facilement les situations redoutées.

\paragraph{Interprétation des résultats : Contrôleur avec débit optimisé}

Lorsqu'on optimise le débit aval, on diminue le nombre de situations redoutées. Elle passent de 551 à 239. Cependant, on peut bloquer le système
car $deadlock = 239$. On peut noter que tous les niveaux critiques sont des situations où l'on se bloque, sinon on aurait $SR > deadlock$ (par l'absurde, si il existe un état dans $NC$ différent de tous ceux de $deadlock$ alors, $|NC \cup deadlock| > |deadlock|$)


\subsubsection{Avec 3 défaillances (1 point)}
\lstinputlisting{Res/System3FCtrl.res}
\lstinputlisting{Res/System3FCtrl3F1I.res}
%\lstinputlisting{Res/System3FCtrl3F2I.res}
%\lstinputlisting{Res/System3FCtrl3F3I.res}
%\lstinputlisting{Res/System3FCtrl3F4I.res}
\paragraph{Interprétation des résultats : Contrôleur sans débit optimisé}
Dans ce contrôleur, il n'est pas possible de se trouver dans un état bloquant ($deadlock = 0$). Il est néanmoins possible de se trouver dans un des nombreux états qui sont des niveaux critiques.
Avec $any\_s = 1832$ et $NC=617$, on peut voir que près d'un tier des états sont des situations critiques.

\paragraph{Interprétation des résultats : Contrôleur avec débit optimisé}
A l'inverse du contrôleur sans optimisation de débit, il n'est pas possible de se retrouver dans des situations ou le niveau de l'eau atteint un niveau critique ($NC = 0$), mais il est possible que le systeme se bloque dans un des 112 états qui sont dans $deadlock$.
Environ un état sur deux sont des états bloquants dans ce système. Il parait donc difficile de contrôler sans jamais se retouver dans un état appartenant à $deadlock$.

\subsection{Calcul des contrôleurs optimisés (2 points)}
% A COMPLETER en expliquant en quoi consiste l'optimisation mise en place.

% A COMPLETER en analysant les contrôleurs optimisés obtenus.
\subsubsection{Avec 0 défaillances}
\lstinputlisting{Res/System0FCtrl0F2I_Opt.res}
\paragraph{Interprétation des résultats : }
En effet, on constate qu'avec le contrôleur optimisé, on a beaucoup moins d'états à gérer par le contrôleur.
Avec $0$ défaillance, on n'a ni des états bloquants, ni des situations redoutées. Donc avec cette version, on peut bien contrôler en évitant toute situation redoutée ou blocante.
Ce contrôleur a les caractéristiques suivantes :
$49$ états au total, et le contrôleur a 9 transitions à partir de son état initial.

\subsubsection{Avec 1 défaillances}
\lstinputlisting{Res/System1FCtrl1F3I_Opt.res}
\paragraph{Interprétation des résultats : }
Ce contrôleur ne peut pas contrôler en évitant toute situation redoutée ou blocante.
Par conséquent, le nombre total d'états a augmenté. De plus, $NC$ partage 36 états avec le $deadlock$, donc on a plus de possibilités de tomber dans des états blocants.

\subsubsection{Avec 2 défaillances}
\lstinputlisting{Res/System2FCtrl2F4I_Opt.res}
\paragraph{Interprétation des résultats :}
Pour ce qui concercne un contrôleur avec $2$ défaillance, on observe qu'on a la même situation(le contrôleur ne peut pas controller en évitant les blocages et les niveaux critiques) qu'on avait eu avec un $1$ défaillance.
Toutfois, si on compare les résultats avec la version non optimisée, on observe qu'on a mois d'états à gérer, en plus les débit d'aval de $out0$, $out1$ et $out2$ sont bien optimisés.

Conclusion : on ne peut pas contrôler sans éviter de tomber dans des états blocants ou dans des situations redoutées même avec l'optimisation.

\subsubsection{Avec 3 défaillances}
\lstinputlisting{Res/System3FCtrl3F3I_Opt.res}
\paragraph{Interprétation des résultats : }
Dans cette version, on voit qu'on ne peut jamais tomber dans des niveaux critiques ($NC$ est à 0), donc on ne peut se retrouver que dans des états blocants ($SR$ ne contient que des états blocants).

De plus, vu qu'on est dans une situation où les 3 valves tombent en panne(3 défaillances), alors il est complétement logique que notre système de cuve va se bloquer.
Concernant les propriétés des transitions, on est sûr que cette version de contrôleur peut nous garantir au départ qu'on aura 16 transitions, à partir lesquelles, on peut aller dans états non redoutées ($CtrCanControl$ = "$16$).
Ainsi que la variable $CCoupGagnantUtile$ nous assure qu'on a $65$ transitions partant depuis des états non $SR$ et qui amènent à des états non $SR$.
\paragraph{Conclusion}
On peut donc déduire qu'avec les contrôleur optimisés, on obtient moins d'états ainsi que moins de transitions à gérer.
De ce fait, on peut mieux s'assure que ces versions de contrôleurs, ont moins de risque de tomber dans des situations redoutées.

\section{Rôle des composants {\tt ValveVirtual} et {\tt CtrlVV} (4 points)}
\paragraph{ValveVirtual : }
Le noeud ValveVirtual sert à simuler un Valve parfaite. Ainsi, lorsqu'une valve physique tombe en panne, la valve virtuelle prend le relais. Ainsi, le contrôleur peut toujours communiquer avec une valve et savoir quel est le niveau d'ouverture de celle ci.

\paragraph{CtrlVV}
Le noeud CtrlVV est un contrôleur similaire à Ctrl, à ceci près qu'il utilise des ValveVirtual à la place des variables de flux $rate$.
Le fait d'utiliser ces ValvesVirtual permet de repérer les disfonctionnement des valves physiques dans le système.

\section{Résultats avec le contrôleur initial {\tt CtrlVV}}
\subsection{Calcul d'un contrôleur}
\subsubsection{Avec 0 défaillance (1 point)}
\lstinputlisting{Res/System0FCtrlVV.res}
\lstinputlisting{Res/System0FCtrlVV0F1I.res}
%\lstinputlisting{Res/System0FCtrlVV0F2I.res}
%\lstinputlisting{Res/System0FCtrlVV0F3I.res}
%\lstinputlisting{Res/System0FCtrlVV0F4I.res}
\paragraph{Interprétation des résultats}
\paragraph{Interprétation des résultats : Contrôleur sans débit optimisé}
Un contrôleur avec des valves virtuelles permet d'avoir un meilleur performance par rapport au nombre de transitions au total.
Ceci nous justifie le fait qu'on obtient moins de transitions à contrôler.

On peut observer que l'utilisation de valve virtuel n'impacte pas les débits ou le nombre de situations redoutées mais seulement le nombre de transitions possibles sur le cas de 0 défaillance.
Le fait qu'on ait ici moins de transitions signifie que le système est plus optimisé et plus rapide.
Dans le cas où les vannes ne sont pas défaillantes, on constate qu’il est possible d’éviter que le système se bloque, ainsi que les situations critiques.
On peut donc déduire qu’une des caractéristiques de ce contrôleur est qu’il est capable de gérer les pannes.


\paragraph{Interprétation des résultats : Contrôleur avec débit optimisé}

\subsubsection{Avec 1 défaillance (1 point)}
\lstinputlisting{Res/System1FCtrlVV.res}
\lstinputlisting{Res/System1FCtrlVV1F1I.res}
%\lstinputlisting{Res/System1FCtrlVV1F2I.res}
%\lstinputlisting{Res/System1FCtrlVV1F3I.res}
%\lstinputlisting{Res/System1FCtrlVV1F4I.res}
\paragraph{Interprétation des résultats}
\paragraph{Interprétation des résultats : Contrôleur sans débit optimisé}
Ce controlleur évite les bloquages, car on a la variable $deadlock = 0$. cependant, il n'est pas possible d'éviter les Niveaux Critiques, qui sont au nombre de 86.
On peut aussi voir que le contrôleur a 20 transitions partant de son état initial.

\paragraph{Interprétation des résultats : Contrôleur avec débit optimisé}
Dans cette version avec les débits optimisés, on ne peut plus se retrouver dans une situation où le niveau de l'eau est critique. Cependant, il est possible de tomber dans un blocages si on arrive dans l'un des 16 états de $deadlock$.
dans ce cas, le nombre d'états bloquant est très faible par rapport au nombre d'états total.

\subsubsection{Avec 2 défaillances (1 point)}
\lstinputlisting{Res/System2FCtrlVV.res}
\lstinputlisting{Res/System2FCtrlVV2F1I.res}
%\lstinputlisting{Res/System2FCtrlVV2F2I.res}
%\lstinputlisting{Res/System2FCtrlVV2F3I.res}
%\lstinputlisting{Res/System2FCtrlVV2F4I.res}
\paragraph{Interprétation des résultats}
\paragraph{Interprétation des résultats : Contrôleur sans débit optimisé}
Comme dans le cas précédent, il n'est pas possible de se retrouver bloquer car $deadlock=0$, mais il est possible d'arriver dans des états ou le niveau de l'eau est trop haut ou trop bas.
Ce controlleur a beaucoup d'états (presque 2400) et a sept fois plus de transitions que d'états.

A partir de l'état initial, le controlleur a le choix entre 26 transitions différentes.
\paragraph{Interprétation des résultats : Contrôleur avec débit optimisé}
Toujours comme dans le cas précédent, dans cette verision avec les débits optimisés, il est possible de tomber dans un $deadlock$ mais il n'est pas possible d'atteindre un niveau critique.

Ce controlleur a beaucoup moins d'état que la version sans optimisation des débits. En effet, il dispose de 274 états, soit près de 9 fois moins que l'autre.
Et il a plus de 21 fois moins de transition que la version non optimisée.


\subsubsection{Avec 3 défaillances (1 point)}
\lstinputlisting{Res/System3FCtrlVV.res}
\lstinputlisting{Res/System3FCtrlVV3F1I.res}
%\lstinputlisting{Res/System3FCtrlVV3F2I.res}
%\lstinputlisting{Res/System3FCtrlVV3F3I.res}
%\lstinputlisting{Res/System3FCtrlVV3F4I.res}
\paragraph{Interprétation des résultats}
\paragraph{Interprétation des résultats : Contrôleur sans débit optimisé}
Ce contrôleur ne permet pas de faire fonctionner le système en assurant qu'il ne tombera pas dans un des niveaux critique d'eau ($NC = 970$).
On peut cependant assurer qu'il ne sera jamais bloquer, car il n'y a aucun état qui fait partit de $deadlock$ car la variable $deadlock=0$.


\paragraph{Interprétation des résultats : Contrôleur avec débit optimisé}
A l'inverse du cas du contrôleur sans les débits optimisé, il est impossible de faire fonctionner le système en assurant qu'il n'y aura pas de blocages car la variable $deadlock = 97$
il n'y a néanmoins pas de possibilité de se retrouver dans une sitation ou les niveaux de l'eau sont critiques.

De plus on peut observer que cette version du contrôleur à plus de douze fois moins d'états au total que la version non optimisée. Mais étant donné que 97 appartiennent à $deadlock$, on se retrouve avec pres de la moitié des états qui sont blocants.
Dans la version précédente, seul un état sur trois étaient des situations redoutées. L'optimisation des débits a donc ajouté, proportionnellement aux états, des situations critiques.

\subsection{Calcul des contrôleurs optimisés (2 points)}
\subsubsection{Avec 0 défaillances}
\lstinputlisting{Res/System0FCtrlVV0F2I_Opt.res}
\paragraph{Interprétation des résultats : }

\subsubsection{Avec 1 défaillances}
\lstinputlisting{Res/System1FCtrlVV1F4I_Opt.res}
\paragraph{Interprétation des résultats : }

\subsubsection{Avec 2 défaillances}
\lstinputlisting{Res/System2FCtrlVV2F3I_Opt.res}
\paragraph{Interprétation des résultats :}


\subsubsection{Avec 3 défaillances}
\lstinputlisting{Res/System3FCtrlVV3F3I_Opt.res}
\paragraph{Interprétation des résultats :}

\section{Conclusion (2 points)}
% A COMPLETER

\end{document}
