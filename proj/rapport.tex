\documentclass[a4paper]{book}
\usepackage{fullpage}

\usepackage[utf8]{inputenc}
\usepackage[T1]{fontenc}
\usepackage[francais]{babel}

\usepackage{latexsym}
\usepackage{fancyhdr}
\usepackage{makeidx}
\usepackage{graphics}
\usepackage{graphicx}
\usepackage{longtable}
\usepackage{moreverb}
\usepackage{listings}

\newcommand{\altarica}{{\sc AltaRica}}

\begin{document}

\title{Master 1, Conceptions Formelles\\
Projet du module \altarica\\
Synthèse (assistée) d'un contrôleur du niveau d'une cuve}

\date{}

\author{Nom1 \and Nom2 \and Nom3}

\maketitle

\chapter{Le sujet}
\input{tank}

\chapter{Le rapport}
\section{Rôle de la constante {\tt nbFailures} (2 points)}
La constante nbFailures est utilisée pour limiter le nombre de configurations atteignables. Elle correspond au nombre de valves qui peuvent être en panne en même temps.

$nbFailures >= (V[0].fail + V[1].fail + V[2].fail)$ correspond au fait que le nombre de valve qui tombent en panne doit toujours être inférieur à la valeur de la constante nbFailure, car cette ligne se trouve dans la section $assert$ du code.

\section{Résultats avec le contrôleur initial {\tt Ctrl}}
\subsection{Calcul d'un contrôleur}
\subsubsection{Avec 0 défaillance (1 point)}
\lstinputlisting{Res/System0FCtrl.res}
\lstinputlisting{Res/System0FCtrl0F1I.res}
%\lstinputlisting{Res/System0FCtrl0F2I.res}
%\lstinputlisting{Res/System0FCtrl0F3I.res}
%\lstinputlisting{Res/System0FCtrl0F4I.res}
\paragraph{Interprétation des résultats : contrôleur non optimisé}
Comme on peut le voir, $deadlock = 0$ indique que le système ne peut pas se retrouver dans un état bloquant.
Le nombre d'états qui correspondent à un niveau critique est $NC = 86$.

Sur l'ensemble des états, il y en à $80$ où l'eau ne sécoule pas ($out0$), $83$ où l'eau s'écoule un peu ($out1$) et 84 où l'eau s'écoule au maximum ($out2$).
Si on fait la somme de ces trois nombres, on tombe sur : $80+84+83 = 247$, qui est bien le nombre total d'états.

\paragraph{Interprétation des résultats : contrôleur optimisé}

\subsubsection{Avec 1 défaillance (1 point)}
\lstinputlisting{Res/System1FCtrl.res}
\lstinputlisting{Res/System1FCtrl1F1I.res}
%\lstinputlisting{Res/System1FCtrl1F2I.res}
%\lstinputlisting{Res/System1FCtrl1F3I.res}
%\lstinputlisting{Res/System1FCtrl1F4I.res}
\paragraph{Interprétation des résultats : contrôleur non optimisé}
Avec 1 défaillance, on remarque qu'on peut pas contrôler en évitant les situations rédoutées car $SR$ est égale à l'union de $deadlock$ et de $NC$.
Donc dans notre cas, on a $SR$ qui partage les mêmes états avec $NC$.

Conclusion : le controleur peut contrôler car la variable CtrlCanControl est positive, mais on ne peut pas éviter de tomber dans des situations redoutées.

\paragraph{Interprétation des résultats : contrôleur non optimisé}
Avec l'optimisation du débit 'aval', on remarque qu'on a moins d'états à gérer donc c'est plus facile pour le contrôleur de controler.

Cependant, on s'apperçoit dorénavant qu'on peut avoir des blocages à cause de la variable $deadlock = 96$, ce qui revient à dire qu'on ne peut pas contrôler sans éviter de tomber dans des situations redoutées car :

$NC$ et $deadlock$ partagent $69$ états et donc on risque d'augmenter les chances de tomber dans des états bloquants.

Conclusion : Malgré le fait qu'on ait optimisé le débit d'aval, on a constaté qu'on ne pourrait pas contrôler sans éviter de tomber dans $SR$ ou dans $deadlock$.
Aussi, on peut bien voir que $CCoupGagnant$ a beaucoup moins de transitions($2909$).

\subsubsection{Avec 2 défaillances (1 point)}
\lstinputlisting{Res/System2FCtrl.res}
\lstinputlisting{Res/System2FCtrl2F1I.res}
%\lstinputlisting{Res/System2FCtrl2F2I.res}
%\lstinputlisting{Res/System2FCtrl2F3I.res}
%\lstinputlisting{Res/System2FCtrl2F4I.res}
\paragraph{Interprétation des résultats}
% A COMPLETER

\subsubsection{Avec 3 défaillances (1 point)}
\lstinputlisting{Res/System3FCtrl.res}
\lstinputlisting{Res/System3FCtrl3F1I.res}
%\lstinputlisting{Res/System3FCtrl3F2I.res}
%\lstinputlisting{Res/System3FCtrl3F3I.res}
%\lstinputlisting{Res/System3FCtrl3F4I.res}
\paragraph{Interprétation des résultats}
% A COMPLETER

\subsection{Calcul des contrôleurs optimisés (2 points)}
% A COMPLETER en expliquant en quoi consiste l'optimisation mise en place.

% A COMPLETER en analysant les contrôleurs optimisés obtenus.

\section{Rôle des composants {\tt ValveVirtual} et {\tt CtrlVV} (4 points)}
% A COMPLETER en expliquant le mécanisme mis en oeuvre et son rôle.

\section{Résultats avec le contrôleur initial {\tt CtrlVV}}
\subsection{Calcul d'un contrôleur}
\subsubsection{Avec 0 défaillance (1 point)}
\lstinputlisting{Res/System0FCtrlVV.res}
\lstinputlisting{Res/System0FCtrlVV0F1I.res}
%\lstinputlisting{Res/System0FCtrlVV0F2I.res}
%\lstinputlisting{Res/System0FCtrlVV0F3I.res}
%\lstinputlisting{Res/System0FCtrlVV0F4I.res}
\paragraph{Interprétation des résultats}
% A COMPLETER

\subsubsection{Avec 1 défaillance (1 point)}
\lstinputlisting{Res/System1FCtrlVV.res}
\lstinputlisting{Res/System1FCtrlVV1F1I.res}
%\lstinputlisting{Res/System1FCtrlVV1F2I.res}
%\lstinputlisting{Res/System1FCtrlVV1F3I.res}
%\lstinputlisting{Res/System1FCtrlVV1F4I.res}
\paragraph{Interprétation des résultats}
% A COMPLETER

\subsubsection{Avec 2 défaillances (1 point)}
\lstinputlisting{Res/System2FCtrlVV.res}
\lstinputlisting{Res/System2FCtrlVV2F1I.res}
%\lstinputlisting{Res/System2FCtrlVV2F2I.res}
%\lstinputlisting{Res/System2FCtrlVV2F3I.res}
%\lstinputlisting{Res/System2FCtrlVV2F4I.res}
\paragraph{Interprétation des résultats}
% A COMPLETER

\subsubsection{Avec 3 défaillances (1 point)}
\lstinputlisting{Res/System3FCtrlVV.res}
\lstinputlisting{Res/System3FCtrlVV3F1I.res}
%\lstinputlisting{Res/System3FCtrlVV3F2I.res}
%\lstinputlisting{Res/System3FCtrlVV3F3I.res}
%\lstinputlisting{Res/System3FCtrlVV3F4I.res}
\paragraph{Interprétation des résultats}
% A COMPLETER

\subsection{Calcul des contrôleurs optimisés (2 points)}
% A COMPLETER en analysant les contrôleurs optimisés obtenus.

\section{Conclusion (2 points)}
% A COMPLETER

\end{document}
